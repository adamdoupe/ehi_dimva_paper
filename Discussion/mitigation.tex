\subsection{Mitigation Strategy}

\label{disc:mitigation}
After demonstrating that \ehi vulnerabilities exist on the web at large, we now describe the most common measures that can be taken to prevent the occurrence of this vulnerability, or at least reduce the impact.

Using a safe, well tested \email library is the preferred way of
preventing \ehi vulnerabilities (removing the burden of input
sanitization from the developer). A list of known secure libraries for
each language and framework discussed is shown in
Table~\ref{tab:maillib}.

%\begin{wraptable}{r}{7cm}
\begin{table}[tbp]
		\centering
		\scriptsize
		\begin{tabular}{|l|l|}
			\hline
			\multicolumn{1}{|c|}{\textbf{Language}} &
			\multicolumn{1}{c|}{\textbf{Mail Libraries}} \\
			\hline
			PHP & {{PEAR Mail\cite{Hagenbuch2016}, PHPMailer\cite{PHPMailer2016}, Swiftmailer\cite{SwiftMailer2016}}}\\
			\hline
			Python & SMTPLib with email.header.Header\\
			\hline
			Java & Apache Commons E-Mail\cite{ACE2016}\\
			\hline
			Ruby & Ruby Mail \textgreater{}= 2.6\cite{RubyMailGem2016}\\
			\hline
		\end{tabular}
		\caption[\titlecap{Mail libraries that prevent e-mail header
            injection}]{Mail libraries that prevent e-mail header
          injection.}
        
		\label{tab:maillib}
\end{table}
%\end{wraptable}

Content management systems such as WordPress and Drupal include
libraries and plugins to prevent \ehi. Thus, websites built with such
CMS' are usually resistant to these attacks. However, it is advised to
use the correct \email plugin, as not all plugins might be secure.

If neither of the two options are feasible (in-house
production, or lack of support infrastructure), developers can
choose to perform proper input sanitization. Sanitization should be
done with RFC5322~\cite{rfc5322} in mind to
ensure that all edge cases are covered.

%% Client Side validation alone is not sufficient, and must be supplemented by server-side validation to mitigate the attack. Constant updates to validation methods are required so that new attack vectors do not harm the website in any way.
%% Test driven development for such validation methods is also encouraged so that we can be reasonably sure of our defense mechanisms.

