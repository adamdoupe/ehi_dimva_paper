\subsection{Lessons Learned}

From our results, it is evident that \ehi vulnerabilities exist in the wild.
% Adam: Sai, I don't understand this math. The 1.46% number is calculated based on the number of forms we were able to fuzz with a malicious payload. This isn't the same as the number of websites. But, using 673/21675680 is not good either because that is # of URLs and it seems like the number you have here is websites. What we need is what % of *websites* (probably estimated as domains) did we find to be vulnerable out of *all* websites/domains that we found. When we use this number, we need to be clear what it is that we are using. -- FIxed this to be clear.
Despite its relatively low occurrence rate compared to the more popular SQL Injection and XSS (Cross-Site Scripting), when we consider total number of domains on the World Wide Web--- 1,018,863,952 according to Internet Live Stats~\cite{InternetLiveStats2016} as of early 2016---and calculate \successWebsitesDelta percent (the occurrence rate of \ehi vulnerability calculated from vulnerable domains as found by our system to total number of domains crawled) of that number, this yields 295,693 domains. Of course, extrapolation in this way is not an accurate measure of the prevalence of \ehi vulnerabilities. However, even with as few as a thousand domains affected by this vulnerability, it can still have a disastrous impact on these domains, and also on overall World Wide Web due to the traffic caused by the sheer number of generated e-mails. 
    
%% We believe that one of the reasons for the small percentage of occurrence (compared to SQL Injection or Cross-Site Scripting), can be attributed to what we like to call the `car parking analogy'.
%%     The car parking analogy is something like this: Imagine that we are parking a car on a road that is prone to attacks by thieves. Now, if all the cars were unlocked, the car that is most likely to get stolen is quite unsurprisingly the most expensive one or the one that is easiest to get away with.
    
%%     Now imagine the same thing on the World Wide Web: we have websites that can each have multiple vulnerabilities. Now, it makes sense for an attacker to try and attack websites with more widespread vulnerabilities such as SQL Injection or XSS, rather than attempt to exploit E-Mail Header Injection, seeing as this requires a more concentrated effort, with carefully crafted payloads and a waiting time for the e-mail to be delivered. SQL Injection attacks and XSS attacks are also better documented, with well-known attack vectors, and automated tools to help detect the presence of these vulnerabilities on websites.
    
%%     This also gives more incentive for the website developer to add protection against attacks such as SQL injection and XSS. The developer might then (possibly with the help of a sanitization library) sanitize the user input and remove \emph{all} special characters, including the newline characters (\textbackslash{}n, \textbackslash{}r), which adversely affects E-Mail Header Injection attacks.

%% 	We come to this conclusion because of our discovery of the \texttt{To header injection}. Clearly, this is possible due to incomplete sanitization performed by the application. We suspect that this incomplete sanitization is actually sanitization that is performed for some other vulnerability, and not specifically for E-Mail Header Injection attacks. We would also like to remark that \texttt{To header injection} is not complete E-Mail Header Injection, but only a special subset.
	
%%     Thus, indirectly, this kind of protection against other attacks affects the attempts to perform E-Mail Header Injection. However, this does not completely negate the attempts if the checks are only on the client-side. Also, even with server-side validation, often, the only input fields that are validated are ones that are either inserted into the database (SQL Injection) and the ones that are displayed to the user as part of the web site (XSS).

% Adam: Please add a citation to a CAPTCHA paper - DONE.
	
	%% This does not mean that the vulnerability is not a large threat. In fact, this vulnerability can also have some major consequences, the least of which can be spamming and phishing attacks.
	%% In today's digital world, identity theft has become all the more common. E-Mail Header Injection provides attackers with the ability to easily extract information about users, not just from a server, but from the user himself, by sending him fake messages that look extremely authentic, since these messages are sent by the mail server of the website itself.
    
    We found two different forms of \ehi: the first one is the traditional one, injecting any header into the \email that allows the attacker complete control over the contents of the \email. 
The second attack has not yet been documented and provides the ability to inject multiple \email addresses into the \texttt{To} field. We call this a \texttt{To header injection}. In this  vulnerability, an attacker can add addresses to the \texttt{To} field of the email with newlines separating the \email addresses. We could not determine if this vulnerability is due to unique flaws in each web application or if this vulnerability is due to an implementation issue with a particular language or framework. However, from our preliminary analysis, it is evident that the vulnerable web applications do not share much in common. 

\texttt{To header injection} allows an attacker to extract information that should be private,
% Adam: It's not clear what this means that we have enough data to spoof the few lines of the message. I thought to header injection just controls the TO field, not the message contents. - Fixed, my bad. DONE.
and in some of these cases, able to inject enough data to spoof other headers of the \email message. From Table~\ref{tab:analysis}, information leakage using \texttt{To header injection} was possible on \ehito forms, while spoofing using \texttt{To header injection} was possible on \ehitoxcheck forms.
    
    %% While not being as impactful as the primary vulnerability, this form of the vulnerability does still provide the ability to send \emails to multiple recipients, and can easily result in information leakage or spam generation on a large scale.
    
